
\section{个人体会}

通过机器学习这门课和这个实验,我对机器学习这个目前热门的大方向有了一个系统性的认识。在过去我因为其他的课程也略微接触到了AI领域的技术,所以在刚接触这个实验的时候,我对于深度学习的任务步骤还是有一个初步的认识,对MNIST, kaggle, CNN等名词比较熟悉,这也让我能够顺利跨过实验的初期。

但是在使用numpy手动实现CNN网络的时候,我发现我对于机器学习实践的细节还是不够熟悉,对于全连接层,卷积层等的前向和反向传播的过程一知半解,遇到了很多困难,好在最后通过查阅文献,阅读pytorch代码等方法,逐渐解决了一个个问题。

在整个模型实现后,我开始逐步调参,从最开始准确度只有不到96,发现模型的反向传播梯度有问题,到尝试使用深度神经网络,却因为使用CPU模型训练太慢而放弃,最后转头选择Lenet-5这个卷积神经网络的开山鼻祖,通过自己的探索和调参取得了不错的结果。

回头来看这次实验,还有很多做得不足的地方可以改进,对于模型,由于最后发现有一些过拟合的现象,我通过添加Dropout来减缓问题,但是其实还以通过旋转,裁剪等方法进行数据集增强,通过这来提高模型的泛化能力。

通过这个实验,我从零开始使用numpy实现了卷积神经网络和一个K邻近算法,然后和我已经很熟悉的预训练模型ResNet50进行对比,让我对与机器学习和神经网络有了一个更加全面的认识,我也锻炼了自己的编程技能和实际问题解决能力,这都让我受益良多。
